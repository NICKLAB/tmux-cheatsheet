\documentclass[landscape]{article}
\usepackage[letterpaper,landscape,margin=1.5cm]{geometry}
\usepackage[table]{xcolor}
\usepackage{dejavu}
\renewcommand*\familydefault{\sfdefault}
\usepackage[T1]{fontenc}
\usepackage{multicol}
\newcommand{\thead}[1]{{\color{black}\bf#1}}

% Tables are colored via xcolor package.  The following command can be
% used at the beginning of a table to ensure that the colors start on
% the right foot.
\newcommand{\blankfirst}{%
  \ifodd\rownum\advance\rownum1\relax\fi}
\newcommand{\coloredfirst}{%
  \blankfirst\advance\rownum1\relax}

\begin{document}
\thispagestyle{empty}
\rowcolors{2}{white}{blue!35}
\begin{center}
  \Huge \texttt{tmux} cheat sheet% 
  \rlap{\color{blue!50}\large{v1.0}}%
  \llap{\raisebox{-1ex}{%
      \normalsize\color{gray}http://www.clintoncurry.net/tmux-cheatsheet}}
\end{center}

\begin{multicols}{3}
  \small
  \section*{Sessions}
  \blankfirst
  \noindent\begin{tabular}{p{0.6in}p{2.3in}} % tblfmt
    \thead{Key} & \thead{Function}\\
    \verb|$| & Rename the session.\\
    \verb|s| & Go to a window in a different session.\\
    \verb|L| & Switch to the last session.\\
    \verb|(| or \verb|)| & Cycle through sessions.\\
    \verb|r| & Refresh the client.\\
    \verb|d| & Detach yourself from the session.\\
    \verb|D| & Detach someone else from the session.\\
  \end{tabular}
  \section*{Misc}
  \blankfirst
  \noindent\begin{tabular}{p{0.6in}p{2.3in}} % tblfmt
    \thead{Key} & \thead{Function}\\
    \verb|C-b| & Send the prefix on through to the current pane's process.\\
    \verb|C-z| & Sends the tmux process to the background.\\
    \verb|:| & Enter a command directly.\\
    \verb|?| & Show a list of all key bindings.\\
    %\texttt{{\raise.17ex\hbox{$\scriptstyle\sim$}}} 
    $\mathtt \sim$ & Show tmux messages you might have missed.\\
    \verb|t| & Show a clock in the current pane\\
    \verb|i| & Displays information about the current window\\
  \end{tabular}  

  \section*{Copy/Paste (Emacs mode)}
  See man page for vi mode commands.\\
  \blankfirst
  \noindent\begin{tabular}{p{0.6in}p{2.3in}} % tblfmt
    \thead{Key} & \thead{Function}\\
    \verb|[| & Enter copy mode\\
    C-Spc & Start selection\\
    C-w & Copy selection\\
    \verb|v| or \verb|R| & Toggle rectangle select\\
    \verb|#| & Past buffer select menu\\
    \verb|]| & Paste into the current buffer.\\
    \verb|-| & Delete the current buffer\\
    PgUp &  Enter copy mode, and scroll up.\\
  \end{tabular}

  \section*{Panes}
  \subsection*{\color{gray!80}Pane navigation and management}
  \blankfirst
  \noindent\begin{tabular}{p{0.6in}p{2.3in}} % tblfmt
    \thead{Key} & \thead{Function}\\
    \verb|o| & Select the next pane\\
    \verb|"| & Splits the area occupied by the current pane into two panes.\\
    \verb|%| & Splits the area occupied by the current pane into two panes.\\
    Arrows & Move to pane in that direction\\
    \verb|C-o| & Rotates the panes around in the current window\\
    \verb|!| & Creates a new window and moves the current pane there.\\
    \verb|q| & Briefly label the panes with their numbers and dimensions\\
    \verb|x| & Kill the current pane\\
    \verb|{| & Swap this pane with the previous one\\
      \verb|}| & Swap this pane with the next one\\
  \end{tabular}
  \subsection*{\color{gray!80}Pane layout}
  \blankfirst
  \noindent\begin{tabular}{p{0.6in}p{2.3in}} % tblfmt
    \thead{Key} & \thead{Function}\\
    \verb|M-1| & even-horizontal\\
    \verb|M-2| & even-vertical\\
    \verb|M-3| & main-horizontal\\
    \verb|M-4| & main-vertical\\
    \verb|M-5| & tiled\\
    Spc & Cycles through the above arrangements.\\
    \verb|C-|arrow & Resize a pane (slowly)\\
    \verb|M-|arrow & Resize a pane (quickly)\\
    \verb|u| & Go backward in this window's layout history (repeatable)\\
    \verb|U| & Go forward in this window's layout history (repeatable)\\
  \end{tabular}   
  \columnbreak

  \section*{Windows}
  \blankfirst
  \noindent\begin{tabular}{p{0.6in}p{2.3in}} % tblfmt\noindent\begin{tabular}{p{.4in}p{0.9in}p{1.5in}} % tblfmt
    \thead{Key} & \thead{Function}\\
    \verb|c| & Create a new window\\
    0\dots9 & Choose window 0, 1, \dots, 9.\\
    \verb|'| & Select a window (prompt for index)\\
    \verb|.| & Give the current window a new index\\
    \verb|f| & Finds a window whose title matches the selected glob.\\
    \verb|,| & Rename the current window\\
    \verb|&| & Kill the current window.\\
    \verb|M-n| & next window that has seen activity\\
    \verb|M-p| & previous window that has seen activity\\
    \verb|l| & Go to the last window\\
    \verb|n| & Go to the next window\\
    \verb|p| & Go to the previous window\\
    \verb|w| & Choose a window from an interactive list.\\
  \end{tabular}
  \section*{On the command line}
  \ifodd\rownum\advance\rownum1\relax\fi
  \coloredfirst
  \begin{tabular}{p{2.9in}}
    \verb|tmux new-session -s session-name|\\
    \verb|tmux list-sessions|\\
    \verb|tmux kill-session -t target|
  \end{tabular}
  \section*{Tips}
  \begin{itemize}
  \item The default prefix is \verb|C-b|.  Some folks change it to \verb|C-a|, but don't you use that one?
  \item After \verb|C-b ?|, use \verb|/| to search for a command;
    \verb|n| and \verb|N| to navigate among results.
  \item Many folks can attach to the same session.
  \item Highly configurable.  Check the man page.
  \item Some keys can be repeated without prefixing again.
  \end{itemize}
\end{multicols}
\end{document}